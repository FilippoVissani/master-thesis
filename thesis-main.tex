\documentclass[12pt,a4paper,openright,twoside]{book}
\usepackage[utf8]{inputenc}
\usepackage{disi-thesis}
\usepackage{code-lstlistings}
\usepackage{notes}
\usepackage{shortcuts}
\usepackage{acronym}

\school{\unibo}
\programme{MSc in Engineering and Computer Science}
\title{Feasibility of Reactive Aggregate Programming via Kotlin Flows}
\author{Filippo Vissani}
\date{\today}
\subject{Laboratory of Software Systems}
\supervisor{Prof. Danilo Pianini}
\cosupervisor{Dott. Gianluca Aguzzi}
\session{IV}
\academicyear{2022-2023}

\mainlinespacing{1.241}

\begin{document}

\frontmatter\frontispiece

\begin{abstract}	
% Max 2000 characters, strict.
% Very brief (e.g. 250-300 words)
% Abstract should briefly point out: Context, Problem/Objectives, Methods/Contribution, Results, Conclusions
\end{abstract}

\begin{dedication} % this is optional
Optional. Max a few lines.
\end{dedication}

\begin{acknowledgements} % this is optional
Optional. Max 1 page.
\end{acknowledgements}

%----------------------------------------------------------------------------------------
\tableofcontents
\listoffigures     % (optional) comment if empty
\lstlistoflistings % (optional) comment if empty
%----------------------------------------------------------------------------------------

\mainmatter

%----------------------------------------------------------------------------------------
\chapter{Introduction}
\label{chap:introduction}
%----------------------------------------------------------------------------------------

% Introduction should set: Context, Scope, Significance (Motivation), Goals/High-level Questions, Methodology (briefly), Organization of the paper (chapters and what they include)

%----------------------------------------------------------------------------------------
\chapter{Background}
\label{chap:background}
%----------------------------------------------------------------------------------------

\section{Functional Programming}

\subsection{Functional Programming in Kotlin}

\section{Reactive Programming}

\subsection{Reactive Programming in Kotlin}

\section{Aggregate Programming}

\subsection{Abstractions}

% Context
% Sensors
% Neighbor
% Path
% Slot
% Export

\subsection{Field Calculus}

\subsection{Proactive Model}

% talk about semantics

\subsection{Reactive Model}

% talk about semantics

%----------------------------------------------------------------------------------------
\chapter{Analysis}
\label{chap:analysis}
%----------------------------------------------------------------------------------------
\section{State of the Art}

\subsection{Protelis}

\subsection{ScaFi}

\subsection{Collektive}

\subsection{FCPP}

\subsection{FRASP}

\section{Design of FRASP}

% talk about constructs and semantics

\section{Design of Collektive}

% talk about constructs and semantics

\section{Integration of FRASP in Collektive}

%----------------------------------------------------------------------------------------
\chapter{Design}
\label{chap:design}
%----------------------------------------------------------------------------------------

\section{Architecture}

\subsection{Purely Reactive Model}

\subsection{Model with Reactive Messages and Sensors}

\section{Detailed Design}

\subsection{Purely Reactive Model}

\subsection{Model with Reactive Messages and Sensors}

%----------------------------------------------------------------------------------------
\chapter{Implementation}
\label{chap:implementation}
%----------------------------------------------------------------------------------------

\section{Purely Reactive Model}

\section{Model with Reactive Messages and Sensors}

\chapter{Evaluation}
\label{chap:evaluation}
%----------------------------------------------------------------------------------------

\section{Analysis of the Ergonomics of the Proposed Models}

\subsection{Purely Reactive Model}

\subsection{Model with Reactive Messages and Sensors}

\section{Testing}

%----------------------------------------------------------------------------------------
\chapter{Conclusion}
\label{chap:conclusion}
%----------------------------------------------------------------------------------------

% Conclusion chapter should point out: (1) Briefly recall problem, starting point and methods adopted, (2) Briefly report Findings, (3) Briefly discuss benefits/limitations, (4) Discuss Future Work

%----------------------------------------------------------------------------------------
% BIBLIOGRAPHY
%----------------------------------------------------------------------------------------

\backmatter

\nocite{*} % comment this to only show the referenced entries from the .bib file

\bibliographystyle{alpha}
\bibliography{bibliography}

\end{document}